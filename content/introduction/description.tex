%The introduction of the Software Requirements Specification (SRS) provides an overview of the entire SRS with purpose, scope, 
%definitions, acronyms, abbreviations, references and overview of the SRS. 

\label{ch:intro}
The aim of this document is to provide in-depth insight of the Ubuntu Touch Audio Mixer software by defining the problem statement in detail. 

It describes expected capabilities from end users while defining high-level product features.


This document was built using references from:


\begin{enumerate}
\item https://www.overleaf.com/latex/templates/cse355-software-requirements-specification-layout/pvjpzxthtngc
\item https://www.perforce.com/blog/alm/how-write-software-requirements-specification-srs-document
\item https://www.geeksforgeeks.org/software-engineering-quality-characteristics-of-a-good-srs/
\item https://www.geeksforgeeks.org/software-engineering-classification-of-software-requirements/?ref=lbp
\item utdallas.edu -SRS4.0 doc
\item https://ieeexplore.ieee.org/document/278253
\end{enumerate}


It was compiled through the LaTeX template found in 

https://www.overleaf.com/latex/templates/cse355-software-requirements-specification-layout/pvjpzxthtngc


It uses Prototype Outline 1 for SRS Section 3 from the IEEE link