\section{Product Scope}


The app will be downloaded by the user through Ubuntu Touch's Open Store. 
It will not be available elsewhere, other than the GitHub repo which hosts source code and possibly test builds of the app.

It is intended to be used in two major ways:

\begin{enumerate}
	\item It can be used as a portable external audio mixer.
	\begin{enumerate} 
		\item This means that the app will act as a bridge between:
			\begin{enumerate}
				\item An external audio playback / microphone device feeding audio input into the phone running the app.
				\item An external audio playback device receiving mixed audio output from the phone running the app.
			\end{enumerate}
		\item This is the main strength of this app for the following reasons: 
		\begin{enumerate}
				\item As of 10/21/2022, there are no Android or iOS alternatives that provide this functionality.
				\item PulseEffects does provide similar functionality, however there is a major limitation: You must run a server on an Linux machine, and have another Linux machine (with the same package installed) connect to this machine in order to receive the mixed audio output
			\end{enumerate}
	\end{enumerate}
	\item It can also be used as an internal audio mixer. 
	\begin{enumerate} 
		\item This means that the app will be able to mix audio across the system.
		\item This functionality is comparable to other apps on the Android and iOS stores, usually with the terms "equalizer", "EQ", "Mixer" in their name.
		\item This functionality leads this app into having desktop counterparts: 
		\begin{enumerate}
			\item EQualizerAPO for Windows (https://sourceforge.net/projects/equalizerapo/)
			\item PulseEffects for Linux (https://github.com/wwmm/easyeffects).
		\end{enumerate}

		\item Why use this Ubuntu Touch implementation instead of Android or iOS? Well:
		\begin{enumerate}
			\item This app will behave closer to its desktop counterparts in that it will not contain ads, subscriptions, and other scummy money-grubbing schemes. If donations are added to support the app, they will stay on the official Open Store page and out of the user's way.
			\item The app is open source, and is not a mysterious black box.
			\item Will be implemented using as much native Ubuntu Touch and Linux functionality as possible.
			\item Will allow the user to branch audio streams, do processing on those branched off streams, and then finally converge them.
			\item The app will allow you to order the processing modules in the order that you want. This allows for more robust audio processing than currently available alternatives.
			\item It will also be written with a faster and more memory efficient programming language compared to Java (Android) or Swift (iOS). This is especially important as real time audio mixing can be a CPU and memory intensive process.
		\end{enumerate}
		
		\item \textbf{NOTE: } This is also the "fall-back" in case functionality \#1 is not possible in the UT environment. However, it will still retain its strength over Android / iOS counterparts.
	\end{enumerate}
\end{enumerate}


\textbf{NOTE:} The above list may be revised over time depending on how closely initial requirements can be fulfilled in the UT environment.
