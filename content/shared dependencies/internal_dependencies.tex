\section{Internal Dependencies}

This section outlines the internal dependencies that the project will have. This includes file I/O, telephone API's, audio stream API's, etc. I think this will mostly be functionality that is already implemented within Ubuntu Touch.

Links (may help)
\begin{itemize}
	\item https://doc.qt.io/archives/qt-5.12/reference-overview.html
	\item https://api-docs.ubports.com/sdk/apps/qml/index.html
	\item https://docs.innerzaurus.com/en/latest/
	\item https://forums.ubports.com/topic/5525/python-examples
	\item https://docs.ubports.com/en/latest/appdev/guides/index.html
	\item https://gitlab.com/TronFortyTwo/parabola
	\begin{itemize}
		\item This project is simlar to what you want to build, and uses the typical C++ and QML workflow. Worth checking out.
		\item The code has only one C++ file.
		\item It also has separate QML files for each screen. It seems that the app is typically entered through the main QML file
		\item QT also has API's for the microphone, audio input/ouput, and other related functions. You do not need to install a sound driver, unlike EqualizerAPO or Voicemeter Banana.
	\end{itemize}
\end{itemize}


TODO: Focus on domain model / UI mockup / functional requirements first. Then, work on outlining the internal libraries / frameworks / tools that UT has implemented that's for functionality.

TODO: Find out where Audio streams are stored / called in Ubuntu Touch. Any limits?

TODO: How to do file I/O?

