% This file contains input statements which comprise the contents of the SRS
% To modify each section of the SRS, simply go to the file that corresponds to that section and make the adjustments.
% The only modifications that should be made to this file should be things like:
%	Adjusting section import order 
%	adding newlines / pagebreaks between sections / chapters
%	Adding new chapters / sections (please follow the current convention)
%	other formatting decisions that are not related to any section




\chapter{Introduction}
%The introduction of the Software Requirements Specification (SRS) provides an overview of the entire SRS with purpose, scope, 
%definitions, acronyms, abbreviations, references and overview of the SRS. 

\label{ch:intro}
The aim of this document is to provide in-depth insight of the Ubuntu Touch Audio Mixer software by defining the problem statement in detail. 

It describes expected capabilities from end users while defining high-level product features.


This document was built using references from:


\begin{enumerate}
\item https://www.overleaf.com/latex/templates/cse355-software-requirements-specification-layout/pvjpzxthtngc
\item https://www.perforce.com/blog/alm/how-write-software-requirements-specification-srs-document
\item https://www.geeksforgeeks.org/software-engineering-quality-characteristics-of-a-good-srs/
\item https://www.geeksforgeeks.org/software-engineering-classification-of-software-requirements/?ref=lbp
\item utdallas.edu -SRS4.0 doc
\item https://ieeexplore.ieee.org/document/278253
\end{enumerate}


It was compiled through the LaTeX template found in 

https://www.overleaf.com/latex/templates/cse355-software-requirements-specification-layout/pvjpzxthtngc


It uses Prototype Outline 1 for SRS Section 3 from the IEEE link

\section{Purpose}

This document is intended to be read by volunteer developers and testers, as well as anyone else curious enough to learn more about the project.

Admittedly, this SRS is more developer/tester oriented since it relies on git version control to track its development. Git is  used instead of maintaining a version history table within the SRS itself; the latter of which is typical for SRS documents created in Microsoft Word or SRS-creation software.

Therefore, learning how to compare git commits and branches will help with understanding the SRS development over time. 


\textbf{NOTE 1:} This document is meant to aid a first official release. It will likely not be maintained once the project has it's official release, and will stay for archival purposes.

\textbf{NOTE 2:} This document is also a loose adaptation of an SRS. As such, it contains design decisions, doesn't make mention of financials, and has some code practices outlined.
\section{Product Scope}


The app will be downloaded by the user through Ubuntu Touch's Open Store. 
It will not be available elsewhere, other than the GitHub repo which hosts source code and possibly test builds of the app.

It is intended to be used in two major ways:

\begin{enumerate}
	\item It can be used as a portable external audio mixer.
	\begin{enumerate} 
		\item This means that the app will act as a bridge between:
			\begin{enumerate}
				\item An external audio playback / microphone device feeding audio input into the phone running the app.
				\item An external audio playback device receiving mixed audio output from the phone running the app.
			\end{enumerate}
		\item This is the main strength of this app for the following reasons: 
		\begin{enumerate}
				\item As of 10/21/2022, there are no Android or iOS alternatives that provide this functionality.
				%TODO: Rewrite this limitation: \item PulseEffects does provide similar functionality, however there is a major limitation: You must run a server on an Linux machine, and have another Linux machine (with the same package installed) connect to this machine in order to receive the mixed audio output
			\end{enumerate}
	\end{enumerate}
	\item It can also be used as an internal audio mixer. 
	\begin{enumerate} 
		\item This means that the app will be able to mix audio across the system.
		\item This functionality is comparable to other apps on the Android and iOS stores, usually with the terms "equalizer", "EQ", "Mixer" in their name.
		\item This functionality leads this app into having desktop counterparts: 
		\begin{enumerate}
			\item EQualizerAPO for Windows (https://sourceforge.net/projects/equalizerapo/)
			\item PulseEffects for Linux (https://github.com/wwmm/easyeffects).
		\end{enumerate}

		\item Why use this Ubuntu Touch implementation instead of Android or iOS? Well:
		\begin{enumerate}
			\item This app will behave closer to its desktop counterparts in that it will not contain ads, subscriptions, and other scummy money-grubbing schemes. If donations are added to support the app, they will stay on the official Open Store page and out of the user's way.
			\item The app is open source, and is not a mysterious black box.
			\item Will be implemented using as much native Ubuntu Touch and Linux functionality as possible.
			\item It will also be written with a faster and more memory efficient programming language compared to Java (Android) or Swift (iOS). This is especially important as real time audio mixing can be a CPU and memory intensive process.
		\end{enumerate}
		
		\item \textbf{NOTE: } This is also the "fall-back" in case functionality \#1 is not possible in the UT environment. However, it will still retain its strength over Android / iOS counterparts.
	\end{enumerate}
\end{enumerate}


\textbf{NOTE:} The above list may be revised over time depending on how closely initial requirements can be fulfilled in the UT environment.

\section{Risk Definition and Management} 

At this early stage within the project, there are no risks. This may change as the project matures.



\chapter{Development Practices}
\label{ch:development}


\section{Code Style}
https://mitcommlab.mit.edu/broad/commkit/coding-and-comment-style/
\section{Test Style}

Will implement a classical unit testing and test driven development style (TDD). I will use the book "Unit Testing Principles, Practices, and Patterns" by Vladimir Khorikov as a guide.

TDD will be implemented using concepts from parts 1 and 2 of the book for unit testing. Part 3 is for integration testing.

By combining TDD with this SRS, I will be able to better outline units of behavior and shared dependencies.

Unfortunately, I think TDD (through unit testing) can only be implemented to a limited degree.

This project is integration test heavy, not unit test heavy. This is because this project has many shared dependencies, and the domain model is fairly simple. Therefore, the TDD unit testing book might have limited applicability. 

For code that can be unit tested, use the "Unit testing" book as a guide for good code design. These tests should actually be in the code, not the SRS. For that reason, I have revised the old functional requirements away from its original unit test design.


\begin{itemize}
\item Link on TDD through classical unit testing:
	\begin{itemize}
	\item https://khalilstemmler.com/articles/test-driven-development/introduction-to-tdd/\#Classic-Inside-OutChicago-and-Mockist-Outside-InLondon-TDD 
	\item Note that this type of TDD (inside out, classical style) will entail applying it to functionality that is specific to the domain model and mocks shared dependencies. Refer to page 37 of unit testing book for details. 
	\end{itemize}
\end{itemize}





\chapter{Functional Requirements}
% They are basically the requirements stated by the user which one can see directly in the final product
% In order to accurately describe the functional requirements, all scenarios must be enumerated.
% https://www.geeksforgeeks.org/software-engineering-classification-of-software-requirements/?ref=lbp
\label{ch: functional}

TODO: These functional requirements were written down in a way that resembled unit tests. I am going to move away from this approach, and rewrite them to be closer to traditional functional requirements. Some of these will be translated to unit tests during the coding stage. That will likely be those that don't focus entirely on a shared dependency.

TODO: Continue adding on to this list. This will be aided through finishing up the UI mockup and having the domain model more clearly understood.

TODO: Import functional requirements files here

\section{User applies mixing effect(s) to audio stream}

\begin{itemize}

	\item The user can choose audio effects from the following categories. This list may grow as the project progresses.
	\begin{itemize}
		\item Equalization
		\begin{itemize}
			\item 15 band EQ
			\item 30 band EQ
			\item variable band EQ
		\end{itemize}

		\item Filtering
		\begin{itemize}
			\item all pass filter
			\item band pass filter
			\item low pass filter
			\item notch filter
			\item high pass filter
			\item high shelf filter
			\item low shelf filter
			\item peaking filter
		\end{itemize}

		\item Miscellaneous
		\begin{itemize}
			\item compressor
			\item limiter
			\item delay
			\item loudness correction
			\item preamp
			\item reverb
			\item panning
			\item polarity flipper
		\end{itemize}

		\item Related open items
		\begin{itemize}
			\item TODO: Add specific parameters to each effect setting
			\item TODO: Any categories to reorganize?
			\item TODO: Some effects here are generic (reverb). Specify the exact effect to be used.
			\item TODO: Decide which, if any, effects will have user customization of parameters. (e.g., filters let the user pick max and min frequencies to cut off at.)
		\end{itemize}
	\end{itemize}
	
	\item Each effect module should contain:
	\begin{enumerate}
		\item Brief description of what the effect does.
		\item Link on where to learn more / tutorial
		\item Brief explanation of parameters and their limits
		\item If licensing makes this applicable: Software that provided the module's capability
	\end{enumerate}
	



\end{itemize}
\section{User connects an effect module's output to a subsequent module's input}


\begin{itemize}
	\item An effect module can only feed output to
	\begin{enumerate}
		\item either one or two effect modules' inputs or
		\item directly to the audio stream destination. This is the case where the module does not connect to any another module.
	\end{enumerate}
	
	%TODO: I may need to adjust this so that the number of connecting states is limited. Basically, I don't want a loop in the connections. See FL Studio patcher image for details.
	\item An effect module's input can only come from the audio stream source, from a previous effect module, or from a converger module.

	\item It is also possible for the audio stream source to be directly connected to the audio stream destination. This means that no audio effects are being applied to the stream whatsoever. 

	\item TODO: Decide if the user can specify more than two modules to connect to, and should there be a hard limit?
	


\end{itemize}



\newpage



\chapter{Non Functional Requirements}
% These are basically the quality constraints that the system must satisfy according to the project contract
% Define how the system should perform 
% https://www.geeksforgeeks.org/software-engineering-classification-of-software-requirements/?ref=lbp
\label{ch: nonfunctional}

TODO: After reviewing external and internal (UT Specific) dependencies and limitations, decide if you want to include this section.



\chapter{Code Design}
\label{ch:design}

\section{UI Mockups}

TODO: Decide if you want to do this in QML or draw IO. You could even use Figma.

TODO: Can be aided through the Domain Modeling / functional requirements
\section{Domain Models}

TODO: Can be aided through the UI Mockups / functional requirements



\chapter{Shared Dependencies}
\label{ch:shared}

This is the chapter that focuses heavily on shared dependencies. 

What are shared dependencies? To paraphrase the book "Unit Testing Principles, Practices, and Patterns" shared dependencies are both the external and out-of-memory dependencies that the project will have. This definition can also include internal dependencies, such as file I/O.

As it stands, this project is heavy on shared dependencies.

Since I will follow the classical approach to TDD, I need to know these shared dependencies so that my unit tests can mock these out.
\section{Internal Dependencies}

This section outlines the internal dependencies that the project will have. This includes file I/O, telephone API's, audio stream API's, etc. I think this will mostly be functionality that is already implemented within Ubuntu Touch.

Links (may help)
\begin{itemize}
	\item https://doc.qt.io/archives/qt-5.12/reference-overview.html
	\item https://api-docs.ubports.com/sdk/apps/qml/index.html
	\item https://docs.innerzaurus.com/en/latest/
	\item https://forums.ubports.com/topic/5525/python-examples
	\item https://docs.ubports.com/en/latest/appdev/guides/index.html
	\item https://gitlab.com/TronFortyTwo/parabola
	\begin{itemize}
		\item This project is simlar to what you want to build, and uses the typical C++ and QML workflow. Worth checking out.
		\item The code has only one C++ file.
		\item It also has separate QML files for each screen. It seems that the app is typically entered through the main QML file
		\item QT also has API's for the microphone, audio input/ouput, and other related functions. You do not need to install a sound driver, unlike EqualizerAPO or Voicemeter Banana.
	\end{itemize}
\end{itemize}


TODO: Focus on domain model / UI mockup / functional requirements first. Then, work on outlining the internal libraries / frameworks / tools that UT has implemented that's for functionality.

TODO: Find out where Audio streams are stored / called in Ubuntu Touch. Any limits?

TODO: How to do file I/O?


\section{Shared Dependencies}

This is the section that outlines external and out-of-memory dependencies that the project will have.

As it stands, this project is heavy on shared dependencies.

Since I will follow the classical approach to TDD, I need to know these shared dependencies so that my unit tests can mock these out.


Below are the results of learning how to import external libraries and linux packages into the project. 


\begin{itemize}
	\item NOTES:
	\begin{itemize}
		\item You can include C++ dependencies (libraries) in your project through clickable.
		\item From what I currently understand, Clickable can compile ARM software into desktop architecture as it uses a container. It may not be able to go the other way (x86 / x86\_64 -> ARM)
		\item Interestingly, it can run the app in a continuous integration environment due to the use of the ARM -> Desktop container. There are guides on how to do CLCI
		\item The consensus on Quora seems to be that the difficulty will range from: rebuilding with an ARM compiler to rewriting from scratch. It all depends on the type of software
	\end{itemize}
\end{itemize}



\begin{itemize}
	\item Links
	\begin{itemize}
		\item https://docs.ubports.com/en/latest/appdev/guides/dependencies.html
		\item https://ubports.com/blog/ubports-news-1/post/introduction-to-clickable-147
		\item https://clickable.bhdouglass.com/en/latest/
		\item https://www.quora.com/How-hard-is-it-to-port-an-application-from-x86-to-ARM
	\end{itemize}
\end{itemize}



\begin{itemize}
\item Contains the largest list of open source audio software that I could find. Look under "Software Development Libraries \& APIs":
	\begin{itemize}
	\item https://github.com/webprofusion/OpenAudio	
	\end{itemize}
\end{itemize}





\textbf{Audio Processing Libraries / Tools:}

%\begin{tabular}{|l|l|l|l|l|l|l|}
\begin{tabular}{|p{2.25cm}|p{2.25cm}|p{2.25cm}|p{2.25cm}|p{2.25cm}|p{2.25cm}|p{2.25cm}|}
\hline
\thead{Name} & \thead{Brief Description} & \thead{License Type} & \thead{ARM Build} & \thead{C++ library} & \thead{Estimated work} & \thead{Link} \\ 
                    \hline
Faust & 
Functional programming language for real-time signal processing & 
GNU General Public License v2.0 & 
Y, doesn't depend on external libraries so compile the generated code into ARM & 
Y, after generating the code & 
May require knowing the full extent of code functionality. That may hamper scalability, and create software design compromises. However, there is a JIT compiler that may help with this. & 
https://faust.grame.fr/ \\    \hline

RustAudio & 
Collection of audio processing and plugin libraries for the Rust language & 
MIT and Apache, to name a few & 
Not sure, may depend on how much of the code is native Rust & 
N, but it is written in Rust & 
This is a collection of REPO's. I think cpal and dsp-chain will take care of most of the use-cases, without too much effort. Unless this can't compile for ARM \( or is too slow\), I might consider this a top pick, pretty straightforward. & 
https://github.com/RustAudio \\    \hline

DISTRHO  & 
C++ framework for creating cross-platform audio plugins. DPF can build for LADSPA, DSSI, LV2, and VST formats. & 
ISC License & 
Most likely, seems to be mostly native code & 
Y & 
Might be a lot, doesn't hold your hand & 
https://github.com/DISTRHO/DPF \\    \hline


JUCE & 
Cross-platform C++ framework for developing desktop and mobile apps and audio plugins & 
\small{https://juce.com/get-juce} & 
Not natively. There are threads for workarounds. Also, most users of this software are for Windows / Mac Desktop users. & 
Y & 
The library itself seems straightforward, but lack of Linux ARM compatibility might add extra work. & 
\small{https://juce.com/learn/tutorials}  \\    \hline



\end{tabular}



\newpage



\chapter{Ubuntu Touch Notes}
\label{ch:UT_notes}

This chapter contains additional notes on items specific to development in the UT environment. There is some decision making that is listed here.


General docs:
https://docs.ubports.com/en/latest/








\section{License}


Choose license from the following list.




\begin{enumerate}
	\item https://www.gnu.org/licenses/gpl-3.0.en.html
	\begin{enumerate}
		\item As far as I can tell, this is the only one with a "viral" approach to software licensing. That is, if you use a GPL componeent in your software project, your entire software project must be GPL.
		\item Will go with this, since I plan to make this an open source, non commerical project (but might open donations in the future)
		\item Most open source linux projects tend to be GPL anyways, and since it is viral it makes sense to just pick this to begin with.
	\end{enumerate}
	\item https://opensource.org/licenses/MIT
	\item https://opensource.org/licenses/BSD-3-Clause
	\item https://opensource.org/licenses/ISC
	\item https://www.apache.org/licenses/LICENSE-2.0
\end{enumerate}


\textbf{Resources:}
\begin{enumerate}
	\item https://www.quora.com/What-are-the-four-licenses-MIT-Apache-GPL-and-Creative-Commons-How-are-they-different-from-one-another-Which-is-a-great-fit-for-open-source
	\item https://www.quora.com/Which-license-should-you-use-on-Github-for-open-source-projects
	\item https://www.gnu.org/licenses/license-recommendations.html
	\item https://www.gnu.org/licenses/why-not-lgpl.html
\end{enumerate}


\section{IDE, Debugger, and Framework Choice}



\begin{itemize}
	\item Packaging, deploying, publishing and tests.
	\begin{itemize}
		\item The Clickable program includes the following in Docker images:
		\begin{itemize}
			\item QtCreator IDE
			\item GDB
			\item Project templates for different languages
			\item Qt 5.12.9
			\item QML API
			\item Suru Icons
		\end{itemize}
		\item Learn more about Clickable:
		\begin{itemize}
			\item https://docs.ubports.com/en/latest/appdev/code-editor.html
			\item https://docs.ubports.com/en/latest/appdev/nativeapp/index.html
			\item https://clickable.bhdouglass.com/en/latest/
		\end{itemize} 
	\end{itemize}	
	\item Language and Frameworks:
	\begin{itemize}
		\item Use either C++ or RUST. This is crucial for low latency performance.
		\item Qt frameworks already providded by the clickable Docker image. This ensures compatability with Ubuntu Touch devices
	\end{itemize}
\end{itemize}

\section{App Suspension}


Apps are suspended whenever not in the foreground, or when the device is locked. When an app is suspended, it cannot receive location data. For this reason, apps will not be able to track your location whenever they are not in use or the device is locked.

\begin{enumerate}
	\item https://docs.ubports.com/en/latest/userguide/dailyuse/location.html
	\item https://docs.ubports.com/\_/downloads/ro/latest/pdf/
\end{enumerate}

\textbf{NOTE:} May need the user to use UT Tweak to remove app suspension for full usage.
However, there may be a way around it.
\section{Misc Links}



Other guides that may be helpful in the future:
\begin{enumerate}
	\item https://docs.ubports.com/en/latest/appdev/guides/index.html
	\item https://phone.docs.ubuntu.com/en/platform/
	\item https://phone.docs.ubuntu.com/en/platform/guides/app-confinement
	\item https://wiki.ubuntu.com/SecurityTeam/Specifications/ApplicationConfinement
	\item https://phone.docs.ubuntu.com/en/platform/guides/lets-talk-about-performance
\end{enumerate}





%\begin{appendices}

%\chapter{Glossary}

\begin{itemize}
	\item Audio Stream Source:
	\item Mixer:
	\item Actively mixed:
	\item Audio Stream Destination:
	\item Effect Module:
	\item Audio Chain:
	\item Previous Effect Module: An effect module that is earlier in the audio chain than the current module. 
	\item Control Module:
	\item Module link:
\end{itemize}	

%\end{appendices}


