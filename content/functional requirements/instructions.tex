
%format for all functional requirements.
%sourced from IEEE SRS Outline, I will follow it loosely
%	In particular, I am more interested in testing units of behavior so I am likely to ignore implementation detail.
%	I will comment those parts of the instructions out.



%List what it is
\textbf{3.1.1 Functional Requirement 1}


%MIGHT NOT USE
\begin{comment}
This subparagraph should provide a description of 
the purpose of the function and the approaches and techniques
employed. It should contain any introductory
or background material which might clarify the
intent of the function
\textbf{3.1.1.1 Introduction}  
\end{comment}


%Pick and choose from here, doesn't need to be a paragraph
This subparagraph should contain:
(a) A detailed description of all data input
to this function to include:
	(i) The sources of the inputs
	(ii) Quantities
	(iii) Units of measure
	(iv) Timing
	(v) The ranges of the valid inputs to include accuracies and tolerances.
(b) The details of operator control requirements should include names and descriptions
of operator actions, and console or operator
positions. For example, this might include
required operator activities such as form alignment - when printing checks.
(c) References to interface specifications or
interface control documents where appropriate.
\textbf{3.1.1.2 Inputs}


%DO NOT USE, ALL IMPLEMENTATION DETAILS
\begin{comment}
This subparagraph should define all of the operations to be performed on
the input data and intermediate parameters to
obtain the output. It includes specification of:
(a) Validity checks on the input data
(b) The exact sequence of operations to in-
clude timing of events
(c) Responses to abnormal situations, for
example:
	(i) Overflow
	(ii) Communication failure
	(iii) Error handling
(d) Parameters affected by the operations
(e) Requirements for degraded operation
(f) Any methods (for example, equations,
mathematical algorithms, and logical opera-
tions) which must be used to transform the sys-
tem inputs into corresponding outputs. For
example, this might specify:
(i) The formula for computing the with-
holding tax in a payroll package
(ii) A least squares curve fitting technique
for a plotting package
(iii) A meteorological model t o be used
for a weather forecasting package
(g) Validity checks on the output data
\textbf{3.1.1.3 Processing} 
\end{comment}


%Pick and choose from here, doesn't need to be a paragraph
Outputs. This subparagraph should con-
tain :
(a) A detailed description of all data out-
put from this function t o include:
	(i) Destinations of the outputs
	(ii) Quantities
	(iii) Units of measure
	(iv) Timing
	(v) The range of the valid outputs is to
include accuracies and tolerances
	(vi) Disposition of illegal values
	(vii) Error messages
(b) References to interface specifications or
interface control documents where appropriate
In addition, for those systems whose require-
ments focus on input/output behavior, the SRS
should specify all of the significant input/out-
put pairs and sequences of pairs. Sequences will
be needed when a system is required t o remem-
ber its behavior so that it can respond to an in-
put based on that input and past behavior; that
is, behave like a finite state machine
\textbf{3.1.1.4 Outputs}




