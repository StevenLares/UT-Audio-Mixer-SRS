% Edit this file to update documentation. 
% Adding a section or chapter automatically modifies the table of contents


%TODO: Add version history to this document.
%This could be a separate document that gets imported here.
%That way, version history can be tracked in Git.


%The introduction of the Software Requirements Specification (SRS) provides an overview of the entire SRS with purpose, scope, 
%definitions, acronyms, abbreviations, references and overview of the SRS. 
\chapter{Introduction}
\label{ch:intro}
The aim of this document is to provide in-depth insight of the Ubuntu Touch Audio Mixer software by defining the problem statement in detail. 

It describes expected capabilities from end users while defining high-level product features.



This document was built using references from


\leavevmode
\newline
https://www.overleaf.com/latex/templates/cse355-software-requirements-specification-layout/pvjpzxthtngc

\leavevmode
\newline
https://www.perforce.com/blog/alm/how-write-software-requirements-specification-srs-document

\leavevmode
\newline
https://www.geeksforgeeks.org/software-engineering-quality-characteristics-of-a-good-srs/

\leavevmode
\newline
https://www.geeksforgeeks.org/software-engineering-classification-of-software-requirements/?ref=lbp

\leavevmode
\newline
utdallas.edu -SRS4.0 doc


\section{Purpose }
%TODO: Keep adjusting
This document is used to "kick off" the project with a strong functional base. 
I intend for this to be open source later on.



\section{Intended Audience}
\section{Intended Use}
The app will be downloaded by the user through Ubuntu Touch's Open Store. 
It will not be available elsewhere, other than the GitHub repo which hosts source code and possibly test builds of the app.

It is intended to be used in two major ways:

\begin{enumerate}
	\item It can be used as a portable external audio mixer.
	\begin{enumerate} 
		\item This means that the app will act as a bridge between:
			\begin{enumerate}
				\item An external audio playback / microphone device feeding audio input into the phone running the app.
				\item An external audio playback device receiving mixed audio output from the phone running the app.
			\end{enumerate}
		\item This is the main strength of this app for the following reasons: 
		\begin{enumerate}
				\item As of 10/21/2022, there are no Android or iOS alternatives that provide this functionality.
				%TODO: Rewrite this limitation: \item PulseEffects does provide similar functionality, however there is a major limitation: You must run a server on an Linux machine, and have another Linux machine (with the same package installed) connect to this machine in order to receive the mixed audio output
			\end{enumerate}
	\end{enumerate}
	\item It can also be used as an internal audio mixer. 
	\begin{enumerate} 
		\item This means that the app will be able to mix audio across the system.
		\item This functionality is comparable to other apps on the Android and iOS stores, usually with the terms "equalizer", "EQ", "Mixer" in their name.
		\item This functionality is also comparable to EQualizerAPO for Windows (https://sourceforge.net/projects/equalizerapo/), or PulseEffects for Linux (https://github.com/wwmm/easyeffects).

		\item Why use this Ubuntu Touch implementation instead of Android or iOS? Well:
		\begin{enumerate}
			\item This app will behave closer to its desktop counterparts in that it will not contain ads, subscriptions, and other scummy money-grubbing schemes. If donations are added to support the app, they will stay on the official Open Store page and out of the user's way.
			\item The app is open source, and is not a mysterious black box.
			\item Will be implemented using as much native Ubuntu Touch and Linux functionality as possible.
			\item It will also be written with a faster and more memory efficient programming language compared to Java (Android) or Swift (iOS). This is especially important as real time audio mixing can be a CPU and memory intensive process.
		\end{enumerate}
		
		\item \textbf{NOTE: } This is also the "fall-back" in case functionality \#1 is not possible in the UT environment. However, it will still retain its strength over Android / iOS counterparts.
	\end{enumerate}
\end{enumerate}


\textbf{NOTE:} The above list may be revised over time depending on how requirements can be fulfilled in the UT environment.


\section{Product Scope}



\section{Risk Definition} 



\chapter{Overall Description}
\label{Overall Description}

\section{User Classes and Characteristics}

\section{User Needs}

\section{Operating Environment}

\section{Constraints}

%Is your project is dependent on any external factors? 
%Are we reusing a bit of software from a previous project? This 
%new project would then depend on that operating correctly and 
%should be included.
\section{Assumptions and Dependencies}

\section{Code Style}
https://mitcommlab.mit.edu/broad/commkit/coding-and-comment-style/


\section{Test Style}

Will also implement a Test Driven development style, using Unit Testing Principles, Practices, and Patterns by Vladimir Khorikov

TDD will likely be implemented using concepts from parts 1 and 2 of the book. Part 3 is on an as-needed basis.

By combining TDD with this SRS, I will be able to better test outline units of behavior and individual requirements (like Ubuntu dependencies separately)



\newpage




\chapter{Requirements}
\label{Requirements}

\section{Functional Requirements}
\section{External Interface Requirements}
\section{System Features}
\section{Non Functional Requirements}





\begin{appendices}
\chapter{Glossary}

\begin{itemize}
	\item Audio Input:
	\item Mixer:
	\item Mixed:
	\item Audio Output:
\end{itemize}	


\end{appendices}


