\section{User applies mixing effect(s) to audio stream}

\begin{itemize}

	\item The user can choose audio effects from the following categories. This list may grow as the project progresses.
	\begin{itemize}
		\item Equalization
		\begin{itemize}
			\item 15 band EQ
			\item 30 band EQ
			\item variable band EQ
		\end{itemize}

		\item Filtering
		\begin{itemize}
			\item all pass filter
			\item band pass filter
			\item low pass filter
			\item notch filter
			\item high pass filter
			\item high shelf filter
			\item low shelf filter
			\item peaking filter
		\end{itemize}

		\item Miscellaneous
		\begin{itemize}
			\item compressor
			\item limiter
			\item delay
			\item loudness correction
			\item preamp
			\item reverb
			\item panning
			\item polarity flipper
		\end{itemize}

		\item Related open items
		\begin{itemize}
			\item TODO: Add specific parameters to each effect setting
			\item TODO: Any categories to reorganize?
			\item TODO: Some effects here are generic (reverb). Specify the exact effect to be used.
			\item TODO: Decide which, if any, effects will have user customization of parameters. (e.g., filters let the user pick max and min frequencies to cut off at.)
		\end{itemize}
	\end{itemize}
	
	\item Each effect module should contain:
	\begin{enumerate}
		\item Brief description of what the effect does.
		\item Link on where to learn more / tutorial
		\item Brief explanation of parameters and their limits
		\item If licensing makes this applicable: Software that provided the module's capability
	\end{enumerate}
	



\end{itemize}